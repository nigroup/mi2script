\documentclass[a4paper,11pt]{article}
\usepackage{a4wide}
\usepackage[utf8]{inputenc}
\usepackage{amssymb}
\usepackage{amsmath}
\usepackage{amsfonts}
\usepackage{enumerate}
\usepackage{fancyhdr}
\usepackage{array}
%\usepackage{../../include/matrices}	
\usepackage{../../include/mathrsfs}
\input{../../machine_intelligence2_script_commands}
\usepackage{graphicx}
\pagestyle{fancy}

% ---------------------------- Headers ----------------------------------------
% Preamble for standalone proofs as used in pure supplementary material.
\lhead{Machine Intelligence II\\Prof. Dr. Obermayer}
\chead{Proof}
\rhead{\\Supplementary Material}
% -----------------------------------------------------------------------------

\begin{document}
\section{Proof of the Cramer-Rao bound}
Normalization of the probability density:
\begin{equation}
	\int d^p \underline{\mathbf{x}} \underbrace{P_{ \big( 
		 \{ \underline{\mathbf{x}}^{(\alpha)} \} 
			; \underline{\mathbf{w}} 
		\big) } }_{\equiv P}= 1
\end{equation}
\begin{equation}
	\begin{array}{ll}
	0 & = \frac{\partial}{\partial \underline{\mathbf{w}}} \int d^p 
		\underline{\mathbf{x}} P\\\\
	& = \int d^p \underline{\mathbf{x}} 
		\frac{\partial P}{\partial \underline{\mathbf{w}}}\\\\
	& = \int d^p P \frac{\partial \ln P}{\partial\underline{\mathbf{w}}}\\\\
	& = \big<\frac{\partial \ln P}{\partial \underline{\mathbf{w}}} \big>_p
	\end{array}
\end{equation}
we then obtain
\begin{equation}
	\begin{array}{ll}
	\Big<\big(\hat{\mathrm{w}}_i - \mathrm{w}_i^* \big) 
		\frac{\partial \ln P}{\partial \mathrm{w}_j} 
		\Big|_{\underline{\mathbf{w}}^*} \Big>_p 
	& = \Big< \hat{\mathrm{w}}_i \frac{ \overbrace{\partial \ln P}^{ =
		\frac{1}{P} \frac{\partial P}{\partial \mathrm{w}_j}} 
		}{\partial \mathrm{w}_j} 
		\Big|_{\underline{\mathbf{w}}^*} \Big>_p - \mathrm{w}_i^* 
			\underbrace{\Big<\frac{\partial \ln P}
				{\partial \mathrm{w}_j}
			\Big|_{\underline{\mathbf{w}}^*} \Big>_p }_{= 0}\\\\
	& = \frac{\partial <\hat{\mathrm{w}}_i>_p}{\partial \mathrm{w}_j}
		\Big|_{\underline{\mathbf{w}}^*}\\\\
	& = \frac{\partial \mathrm{w}_i}{\partial \mathrm{w}_j} 
		\Big|_{\underline{\mathbf{w}}^*}\\
	& \uparrow \text{estimator without bias}\\\\
	& = \delta_{ij}
	\end{array}
\end{equation}
let $\underline{\mathbf{a}}$ and $\underline{\mathbf{b}}$ be arbitrary vectors, then:
\begin{equation}
	\Big< \underline{\mathbf{a}}^T \big(\hat{\underline{\mathbf{w}}} -
		\underline{\mathbf{w}}^* \big) 
		\Big( \frac{\partial \ln P}{\partial \underline{\mathbf{w}}}
		\Big)^T \underline{\mathbf{b}}
	\Big>\Big|_{\underline{\mathbf{w}}^*}
	= \underline{\mathbf{a}}^T \underline{\mathbf{b}}
\end{equation}
Application of the Cauchy-Schwarz inequality:
\begin{equation}
	\bigg\{ \int f_{(\underline{\mathbf{D}})} g_{(\underline{\mathbf{D}})}
		h_{(\underline{\mathbf{D}})} d \underline{\mathbf{D}}
	\bigg\}^2 \leq \bigg\{
		\int f_{(\underline{\mathbf{D}})} g_{(\underline{\mathbf{D}})}^2
		d \underline{\mathbf{D}}
	\bigg\} \bigg\{	
		\int f_{(\underline{\mathbf{D}})} h_{(\underline{\mathbf{D}})}^2
		d \underline{\mathbf{D}}
	\bigg\}
\end{equation}
with:
\[ \underline{\mathbf{D}} = \big\{\underline{\mathbf{x}}^{(\alpha)}\big\}
\]
\[ f_{(\underline{\mathbf{D}})} = P_{\big(\{\underline{\mathbf{x}}^{(\alpha)}\}
	, \underline{\mathbf{w}} \big)}
\]
\[ g_{(\underline{\mathbf{D}})} = \underline{\mathbf{a}}^T \big(
	\hat{\underline{\mathbf{w}}} - \underline{\mathbf{w}}^* \big)
\]
\[ h_{(\underline{\mathbf{D}})} = \Bigg(\frac{\partial \ln 
		P_{\big(\{\underline{\mathbf{x}}^{(\alpha)}\}\big)}}{
			\partial \underline{\mathbf{w}}}
	\Bigg)^T \underline{\mathbf{b}}
\]
yields:
\begin{equation}	
	\big( \underline{\mathbf{a}}^T \underline{\mathbf{b}} \big)^2
	\leq \underline{\mathbf{a}}^T \Big< \big( \hat{\underline{\mathbf{w}}}
		- \underline{\mathbf{w}}^* \big) 
	\big( \hat{\underline{\mathbf{w}}} - \underline{\mathbf{w}}^* \big)^T
	\Big>_p \underline{\mathbf{a}} \underline{\mathbf{b}}^T 
	\Big< \frac{\partial \ln P}{\partial \underline{\mathbf{w}}}
	  \Big( \frac{\partial \ln P}{\partial \underline{\mathbf{w}}} \Big)^T
	\Big>_p\bigg|_{\underline{\mathbf{w}}^*} \underline{\mathbf{b}}
\end{equation}
Using:
\begin{equation}
	\begin{array}{ll}
	\Big< \frac{\partial^2 \ln P}{\partial \mathrm{w}_i 
					\partial \mathrm{w}_j}
	\Big>_p 
	& = \Big< \frac{\partial}{\partial \mathrm{w}_i} 
		\Big( \frac{1}{P} \frac{\partial P}{\partial \mathrm{w}_i}
		\Big) \Big>_p \\\\
	& = \Big< \Big\{-\frac{1}{P^2} \frac{\partial P}{\partial \mathrm{w}_i}
		\frac{\partial P}{\partial \mathrm{w}_j} + \frac{1}{P} 
		\frac{\partial}{\partial \mathrm{w}_i} \frac{\partial P}{
			\partial \mathrm{w}_j} \Big\} \Big>_p\\\\
	& = -\Big< \frac{\partial \ln P}{\partial \mathrm{w}_i} 
		\frac{\partial \ln P}{\partial \mathrm{w}_j}
		\Big>_p + \Big< \frac{1}{P}
		\frac{\partial}{\partial \mathrm{w}_i} 
			\Big(P \frac{\partial \ln P}{\partial \mathrm{w}_j}
			\big) \Big>_p\\\\
	& = -\Big< \frac{\partial \ln P}{\partial \mathrm{w}_i} 
		\frac{\partial \ln P}{\partial \mathrm{w}_j} \Big>_p 
		+ \frac{\partial}{\partial \mathrm{w}_i} 
		\underbrace{\Big< \frac{\partial \ln P}{\partial 
					\mathrm{w}_j} \Big>_p}_{= 0}
	\end{array}
\end{equation}
we obtain:
\begin{equation}
	\big( \underline{\mathbf{a}}^T \underline{\mathbf{b}} \big)^2
	\leq \big( \underline{\mathbf{a}}^T 
		\underline{\sum} \underline{\mathbf{a}} \big)
		\big( \underline{\mathbf{b}}^T \underline{\mathbf{M}}
			\underline{\mathbf{b}} \big)
\end{equation}
let $\underline{\mathbf{b}} = \underline{\mathbf{M}}^{-1} 
\underline{\mathbf{a}}$ (ok, because $\underline{\mathbf{b}}$ can be an arbitrary vector), then:
\begin{equation}
	\big( \underline{\mathbf{a}}^T \underline{\mathbf{M}}^{-1} 
		\underline{\mathbf{a}} \big)^2 
	\leq \big( \underline{\mathbf{a}}^T \underline{\sum} 
		\underline{\mathbf{a}} \big)
		\big( \underline{\mathbf{b}}^T \underline{\mathbf{M}}^{-1} 
		\underline{\mathbf{b}} \big)
\end{equation}
\begin{equation}
	\big( \underline{\mathbf{a}}^T \underline{\mathbf{M}}^{-1} 
		\underline{\mathbf{a}} \big)
	\leq \big( \underline{\mathbf{a}}^T \underline{\sum}
		\underline{\mathbf{a}} \big) \text{ for all vectors 
			$\underline{\mathbf{a}}$}
\end{equation}
Consequence:
\begin{equation}
	\underline{\sum} - \underline{\mathbf{M}}^{-1} 
		\text{ is positive semidefinite}
\end{equation}
\end{document}
