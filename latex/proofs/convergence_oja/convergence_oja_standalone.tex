\documentclass[a4paper,11pt]{article}
\usepackage{a4wide}
\usepackage[utf8]{inputenc}
\usepackage{amssymb}
\usepackage{amsmath}
\usepackage{amsfonts}
\usepackage{enumerate}
\usepackage{fancyhdr}
\usepackage{array}
%\usepackage{../../include/matrices}	
\usepackage{../../include/mathrsfs}
\input{../../machine_intelligence2_script_commands}
\usepackage{graphicx}
\pagestyle{fancy}

% ---------------------------- Headers ----------------------------------------
% Preamble for standalone proofs as used in pure supplementary material.
\lhead{Machine Intelligence II\\Prof. Dr. Obermayer}
\chead{Proof}
\rhead{\\Supplementary Material}
% -----------------------------------------------------------------------------

\begin{document}
\section{Convergence Properties of Oja's Rule}
\textcircled{1} small learning steps $\leadsto$ average over all patterns
\begin{equation}
	\begin{array}{ll}
	\Delta \mathrm{w}_j 
		& = \frac{\epsilon}{p} \sum\limits_{\alpha = 1}^p
			\bigg\{ \sum\limits_{k = 1}^N \mathrm{w}_k 
			\mathrm{x}_k^{(\alpha)} \mathrm{x}_j^{(\alpha)}
			- \mathrm{w}_j \sum\limits_{k,l = 1}^N
			\mathrm{w}_k \mathrm{w}_l 
			\mathrm{x}_k^{(\alpha)} \mathrm{x}_l^{(\alpha)}
			\bigg\} \\\\
		& = \epsilon \bigg\{ \sum\limits_{k = 1}^N 
			\mathrm{w}_k C_{kj} - \mathrm{w}_j
			\sum\limits_{k,l = 1}^N \mathrm{w}_k \mathrm{w}_l
			C_{kl}
			\bigg\} \\\\
	\Delta \vec{w} & = \epsilon \Big\{ 
		\underbrace{ \vec{C} \vec{w} }_{
			\substack{\text{Hebbian} \\ \text{rule}}}
		- \underbrace{ 
			\underbrace{ \big( \vec{w}^T \vec{C} \vec{w} \big) }_{
			\substack{\text{always} \\ \text{positive}}}
			\vec{w} }_{\text{decay term}} \Big\}
	\end{array}
\end{equation}
\textcircled{2} stationary states $\vec{w}^*$ of Oja's rule \\
\indent$\corresponds$ normalized eigenvectors $\vec{e}_j$ of the correlation matrix $\vec{C}$\\\\
Proof:
\[ \begin{array}{ll}
	\text{stationary state:} 
	& \Delta \vec{w} \eqexcl \vec{0} \\\\
	\text{ansatz:} 
	& \vec{w}^* = C \vec{e}_j \sim  \vec{e}_j
\end{array} \]
insertion into Oja's rule:
\begin{equation}
	\begin{array}{l}
	\Delta \vec{w} = \epsilon \Big\{ C \lambda_j \vec{e}_j
		- C^3 \lambda_j \vec{e}_j 
		\Big\} \eqexcl O \\\\
	\Rightarrow C^2 = 1 \\\\
	\Rightarrow \vec{w}^* = \pm \vec{e}_j
	\end{array}
\end{equation}
\textcircled{3} The stationary state $\vec{w}^* = \vec{e}_j$ is stable if and only if $\vec{e}_j = \vec{e}_1$, i.e. if $\vec{e}_j$ is the eigenvactor with the largest eigenvalue.\\\\
Proof: linear stability analysis
\[ \begin{array}{l}
	\lambda_1 > \lambda_2 > \ldots > \lambda_N \text{ eigenvalues of } 
		\vec{C} \\\\
	\vec{w} = \vec{e}_j + \vec{\eta} \leftarrow \text{ small deviation
		from the stationary state}
\end{array} \]
\begin{equation}
	\begin{array}{lll}
	\Delta \vec{w} =
	& \Delta \vec{\eta} 
	& = \epsilon \vec{C} (\vec{e}_j + \vec{\eta}) - \epsilon \Big\{
		(\vec{e}_j + \vec{\eta})^T \vec{C} (\vec{e}_j + \vec{\eta})
		\Big\} (\vec{e}_j + \vec{\eta}) \\\\
	&& = \epsilon \Big\{ \lambda_j \vec{e}_j + \vec{C} \vec{\eta} 
		- \lambda_j \vec{e}_j - \lambda_j \vec{\eta} 
		- 2 \lambda_j (\vec{e}_j^T \vec{\eta}) (\vec{e}_j + \vec{\eta})
		- (\vec{\eta}^T \vec{C} \vec{\eta})(\vec{e}_j + \vec{\eta})
		\Big\} \\\\
	& \Delta{\eta} 
	& = \epsilon \Big\{ - 2 \lambda_j (\vec{e}_j^T \vec{\eta}) \vec{e}_j
		+ \vec{C} \vec{\eta} - \lambda_j \vec{\eta}
		\Big\} + \underbrace{ O_{(\vec{\eta}^2)} }_{
			\substack{ \text{discarded for} \\ |\vec{\eta}| 
				\rightarrow 0}}
	\end{array}
\end{equation}
Projection onto the eigenvectors $\vec{e}_k$:
\begin{equation}
	\vec{e}_k^T \Delta \vec{\eta} = \epsilon \Big\{
		(\lambda_k - \lambda_j) \vec{e}_k^T \vec{\eta} 
		- 2 \lambda_k (\vec{e}_k^T \vec{\eta}) 
		\underbrace{ \delta_{kj} }_{\substack{
			\text{Kronecker-} \\ \text{Delta}}}
		\Big\}
\end{equation}
\underline{case I:} $k = j$
\begin{equation}
	\begin{array}{l}
		\vec{e}_k^T \Delta \vec{\eta} = 
		\underbrace{ -2 \epsilon \lambda_k}_{
			\substack{\text{always} \\ \text{negative}}}
		\vec{e}_k^T \vec{\eta} \\\\
		\Rightarrow \vec{e}_k^T \vec{\eta} \rightarrow 0
	\end{array}
\end{equation}
\underline{case II:} $k \neq j$
\begin{equation}
	\begin{array}{l}
	\vec{e}_k^T \Delta \vec{\eta} = 
	\underbrace{ \epsilon (\lambda_k - \lambda_j) }_{
		\substack{\text{factor is negative} \\
			\text{only if } \lambda_j \text{ is} \\
			\text{the largest eigenvalue}}} 
		\vec{e}_k^T \vec{\eta} \\\\
	\Rightarrow \text{ if } \lambda_j = \lambda_1, \text{ then }
		\vec{e}_k^T \vec{\eta} \rightarrow 0
	\end{array}
\end{equation}
\end{document}
